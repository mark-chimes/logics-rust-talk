\documentclass{beamer}
\usepackage{listings, listings-rust}
\usepackage{graphicx}

\lstset{escapeinside={<@}{@>}}


\usepackage{beamerthemeTUW}
%\usepackage{enumitem}
\setbeamercovered{transparent}


\title{Formally Verifying Rust}
\subtitle{Making sure your Rust code works}
\author{Mark Chimes}
\date{\today}

\begin{document}


\begin{frame}
    \titlepage
\end{frame}
%
%\begin{frame}{Outline}
%\tableofcontents
%\end{frame}

\section{Rust}
\begin{frame}{Overview}
\begin{block}{What will we do?}
\begin{itemize} 
	\item Introduce Rust
	\item Intrduce Formal verification
	\item Motivate why formally verify with Rust
	\item Examples of Rust Verification
\end{itemize}
\end{block}
\end{frame}


\section{Rust Intro}
\begin{frame}{Rust}
\begin{block}{What is Rust?}
\begin{itemize}
	\item Rust is a \textbf{systems} programming language
	\begin{itemize}
		\item low-level control
		\item fast / efficient
	\end{itemize}
	\item strongly-typed
	\item ``ownership" and ``borrowing"
\end{itemize}
\end{block}
\begin{block}{Special Sauce} 
Memory safety guarantees without garbage collection
\end{block}
\end{frame}


\begin{frame}[fragile]
\frametitle{Ownership and Borrow Examples}
\begin{lstlisting}[language=Rust]
foo(s); // foo takes ownership of s
bar(&t); // bar borrows t as read-only
baz(&mut u); // baz borrows u and can modify it
\end{lstlisting}
\end{frame}


\begin{frame}{Rust: Variable Scopes}
\begin{block}{} 
\begin{itemize}
\item Rust automatically calculates lifetimes
\item When an object goes out of scope, its resources are automatically freed
\item ``Resource Acquisition is Initialization" 
\end{itemize}
\end{block}
\begin{quote}
``resource is guaranteed to be held between when initialization finishes and finalization starts 
and to be held only when the object is alive"
\end{quote}
\end{frame}


\begin{frame}[fragile]
\frametitle{Drop example}

\begin{lstlisting}[language=Rust]
fn borrow_box(ibox: &Box<i32>)
\end{lstlisting}

\begin{block}{}
\begin{lstlisting}[language=Rust]
fn create_box() {
    let my_box = Box::new(1i32);
    borrow_box(&my_box);

    // my_box memory freed here
}
\end{lstlisting}

\end{block}

\end{frame}


\begin{frame}{Borrowing}
\begin{block}{Rust Ownership and Borrowing}
\begin{itemize} 
	\item One owner
	\item One read/write borrow
	\item Multiple read-only borrows
	\item When lifetime ends, item is returned
\end{itemize} 
\end{block}
\end{frame}

\begin{frame}[fragile]
\frametitle{Multiple Borrows Example}
\begin{lstlisting}[language=Rust]
fn calculate_length(s : &String) -> usize
\end{lstlisting}

\begin{block}{}
\begin{lstlisting}[language=Rust]
    let mut s = String::from("hello");
    
    let t1: &mut String = &mut s;
    // <@\textcolor{blue}{t1 never used again, gets dropped}@>
    let t2: &mut String = &mut s;

    let len = calculate_length(t2);
 \end{lstlisting}
\end{block}

\end{frame}


\begin{frame}[fragile]
\frametitle{Not Fine}
\begin{lstlisting}[language=Rust]
fn calculate_length(s : &String) -> usize
\end{lstlisting}

\begin{block}{}
\begin{lstlisting}[language=Rust]
    let mut s = String::from("hello");
    
    let t1: &mut String = &mut s;
    // <@\textcolor{red}{t1 used again later, still active}@>
    let t2: &mut String = &mut s;

    let len = calculate_length(<@\textcolor{red}{t1}@>); // fails
 \end{lstlisting}
\end{block}

\end{frame}




\begin{frame}{Unsafe Rust}
\begin{block}{Unsafe Rust}
\begin{itemize} 
	\item sections of code can be tagged `unsafe' - indicates borrow checker does not apply
	\item necessary part of Rust - e.g. de/allocation optimizations, cyclic references, call external code
	\item unsafe code should be wrapped in a safe API
	\item lots of the Rust standard library uses `unsafe'
	\item you can usually find a standard safely wrapped version of whatever unsafe thing you want to do
	\item provides a challenge for formal verification
\end{itemize} 
\end{block}
\end{frame}


\section{Formal Verification Intro}
\begin{frame}{Formal Verification}
\begin{block}{What is Formal Verification?}
\begin{itemize} 
	\item Program verification: Logically ``prove" correctness of a program
	\item labour-intensive process
	\item Can involve manual encoding in e.g. Coq (Rocq), Agda, Lean, Idris
	\item Some tools for (partial) automation: Dafny, Viper, etc.
	\item Often rely on an SMT solver and/or Horn Clause solver
	\item ``very high level of reliability" BUT only as good as your mathematical model 
	\item Other methods (e.g. model checking) also exist.
\end{itemize} 
\end{block}
\end{frame}


\begin{frame}
\includegraphics[scale=0.6]{pictures/verification/bubblesort.png}
\end{frame}

\begin{frame}
\begin{center}
\includegraphics[scale=0.3]{pictures/verification/bubblesort_annotated.png}
\end{center}
\end{frame}

\section{Combination}
\begin{frame}{}
\begin{center}
What makes Rust interesting for Verification?
\end{center}
\end{frame}


\begin{frame}[fragile]{Aliasing Problem}
Two pointers
\begin{block}{C program}
\begin{lstlisting}[language=C]
void increment_both(int *x, int *y)
{
    *x = *x + 1; 
    *y = *y + 1;   
}
 \end{lstlisting}
\end{block}

Want to prove after ``increment both" is done, that\\
x\_end = x\_start + 1 and \\
y\_end = y\_start + 1

\end{frame}

\begin{frame}[fragile]{Aliasing Problem}
\begin{block}{C program}
\begin{lstlisting}[language=C]
void increment_both(int *x, int *y)
{
    *x = *x + 1; 
    *y = *y + 1;   
}
 \end{lstlisting}
\end{block}

\begin{block}{Calling ``increment both" with the same variable twice}
\begin{lstlisting}[language=C]
int a = 0;
increment_both(&a, &a); 
\end{lstlisting}
\end{block}
\end{frame}

\begin{frame}{Proof Difficulty}
\begin{itemize}
\item
Aliasing could happen anywhere in the program
\item
Correct usage of increment\_both relies on \emph{non-local} information!
\item
Strategies exist, involving e.g. separation logic
\item But in Safe Rust, mutable references can \emph{never alias}
\item Rust also strongly typed, and borrows must be marked read / write
\end{itemize} 
\end{frame}



\begin{frame}[fragile]
\vspace{3em}
\begin{block}{C program}
\begin{lstlisting}[language=C]
const char *msg = "hello, world!"; 
int len = get_length(msg);
// do something else with msg
\end{lstlisting}
\end{block}

\begin{block}{Rust program}
\begin{lstlisting}[language=rust]
let msg = "hello, world!"; 
let len = get_length(&msg);  
// do something else with msg
\end{lstlisting}
\end{block}


in C, must add conditions that get\_length does not modify 
msg

but in Rust, msg is not mutable
\end{frame}

\begin{frame}{Unsafe Rust and Verification} 
\begin{itemize}
\item 
Unsafe Rust is essential for functioning of Rust 
\item 
It is more difficult to verify since we do not have the safety guarantees \emph{inside} unsafe blocks
\item 
But we can still isolate (un)safety, make assumptions, prove separately
\item 
Some tools for unsafe Rust exist
\end{itemize} 
\end{frame} 




\section{Rust Verifiers}

\begin{frame}{Rust Verification Tools}
\begin{itemize} 
\item Prusti (Viper)
\item Flux
\item Creusot
\item RustHorn and RustHornBelt
\item Aeneas
\item Gilian-Rust
\end{itemize} 
\end{frame}

\begin{frame}{Prusti / Viper} 
\begin{center}
\includegraphics[scale=0.3]{pictures/viper.png}
\end{center}
\end{frame}


\begin{frame}{Prusti / Viper} 
\begin{itemize}
\item Prusti built on top of Viper
\item Automatically check for panics and e.g. integer overflow
\item Contract-based specifications
\item Viper and Prusti uses a separation logic: currently Prusti re-encodes Rust memory model in separation logic during translation to Viper (might potentially use Place Capability Graphs in future)
\end{itemize}
\end{frame}

\begin{frame}{Flux} 
\begin{itemize}
\item Refinement Type Checking (liquid types) - statements to be proven are encoded as types
\item Runs as a Rust compiler plugin 
\item Less powerful than Prusti but much faster and with fewer annotations
\item Concept of "strong mutability / structural mutability" - e.g. whether a vector's length can change or just the values of its elements
\item No quantified statements - cannot verify e.g. sortedness of list
\end{itemize}
\end{frame}

\begin{frame}{Creusot} 
\begin{itemize}
\item ``
Creusot is a deductive verifier for Rust code. It verifies your code is safe from panics, overflows, and assertion failures. By adding annotations you can take it further and verify your code does the correct thing.
"
\item Translate Rust to Coma / Why3
\item CreuSAT a SAT-solver built in Rust and verified with Creusot
\end{itemize}
\end{frame}

\begin{frame}{RustHorn} 
\begin{block}{RustHorn}
\begin{itemize}
\item Reduces Rust to Constrained Horn Clauses
\item ``Semantic typing"
\end{itemize}
\end{block}
\begin{block}{RustHornBelt}
\begin{itemize}
\item Mechanized translation into Rocq / Coq
\item ``
first machine-checked proof of soundness for RustHorn-style verification
"
\item Reduces Rust to Constrained Horn Clauses and then solves with CHC solvers
\item extends RustBelt’s model to reason not only about safety but also functional correctness
\item soundness proofs for APIs implemented with unsafe
\end{itemize}
\end{block}

\end{frame}

\begin{frame} {Aeneas}

\begin{itemize}
\item 
 verification toolchain
\item
Translates Rust’s MIR to $\lambda$-calculus
\item Uses Charon which compiles Rust programs (MIR) to an intermediate representation called LLBC
\item
backends for F\*, Rocq / Coq, HOL4 and LEAN.
\end{itemize}
\end{frame} 



\begin{frame} {Gilian-Rust}

\begin{itemize}
\item 
Attempt to handle unsafe Rust
\item 
no real tooling available for the public
\end{itemize}


\begin{center}
\includegraphics[scale=0.2]{pictures/gilian-creusot}
\end{center}

\end{frame} 





\begin{frame} {Future Work}

\begin{itemize}
\item Verification of data-structures of specific shapes - e.g. Doubly-Linked Lists, Red-Black Trees
\item Verification a la refinement types
\item Possibly expand Flux to allow some correctness proofs
\item Look into alternatives for modelling Rust memory for use with tools
\end{itemize}

\end{frame}

\begin{frame}{EU Flag} 
\begin{center}
\includegraphics[scale=0.1]{pictures/euflag.png}

The project leading to this application has received funding from the
European Union's Horizon 2020 research and innovation programme under grant agreement No
101034440.
\end{center}
\end{frame}






\end{document}